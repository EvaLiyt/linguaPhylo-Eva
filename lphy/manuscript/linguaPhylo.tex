\documentclass[11pt]{article}

\usepackage{minted}

\begin{document}

\title{Lingua Phylo: a model specification language for communicating and reproducing probabilistic models for phylogenetic analysis}
\author{Alexei J. Drummond}
\date{}
\maketitle


\section{Introduction}

A new paradigm for scientific computing and data science has begun to emerged in the last decade. A recent example is the publication of the first "computationally reproducible article" using eLife's Reproducible Document Stack which blends features of a traditional manuscript with live code, data and interactive figures.

Although standard tools for statistical phylogenetics provide a degree of reproducibility and reusability through popular open-source software and computer-readable data file formats, there is still much to do. The ability to construct and accurately communicate probabilistic models in phylogenetics is frustratingly underdeveloped. There is low interoperability between different inference packages (e.g. BEAST1, BEAST2, MrBayes, RevBayes), and the file formats that these software use have low readability for researchers.

In this paper we describe two related project, LinguaPhylo (LPhy for short) and LPhyBEAST. 



\section{ LinguaPhylo}

LinguaPhylo is a model specification language to concisely and precisely define probabilistic phylogenetic models. The aim is to work towards a {\it lingua franca} for probabilistic models of phylogenetic evolution. This language should be readable by both humans and computers. Here is an example:

\begin{minted}{Jags}
lambda ~ LogNormal(meanlog=3.0, sdlog=1.0);
tree ~ Yule(birthRate=lambda, n=16);
D ~ PhyloCTMC(L=200, Q=jukesCantor(), tree=tree);
\end{minted}

Each of the lines in this specification expresses how a random variable (to the left of the tilde) is generated by a generative distribution to the right.

The first line creates a random variable (\texttt{lambda}), that is log-normally distributed. The second line creates a tree (\texttt{tree}) with 16 taxa from the Yule process with a lineage birth rate equal to \texttt{lambda}. The third line produces a multiple sequence alignment (\texttt{D}) with a length of 200, by simulating a Jukes Cantor model of sequence evolution down the branches of  \texttt{tree}. As you can see, each of the random variables depends on the previous, so this is a hierarchical model that ultimately defines a probability distribution over sequence alignments of size $16 \times 200$.

\subsection{Tree generative distributions}

There are many statistical programming languages such as JAGS, BUGS and Stan that provide the possibility of succinctly describing statistical models. The unique feature of phylogenetic analysis the phylogenetic tree. This is a complex high-dimensional object, part discrete, part continuous. There is no bijection between tree space and Euclidean space, so it can not be treated with standard statistical distributions. As a result specialist software is needed to perform inference.

The aim of LPhy is describe the standard phylogenetic tree distributions succinctly and precisely. 

\subsection{Models of evolutionary rates and sequence evolution}

Another key distribution necessary to perform phylogenetic inference is the phylogenetic continuous-time Markov process and its inference equivalent the phylogenetic likelihood. 




\subsection{LinguaPhylo Studio}

Along with the language definition, we also provide software to specify and visualise models as well as simulate data from models defined in LPhy. 

This software will also provide the ability for models specified in the LPhy language to be applied to data using standard inference tools such as MrBayes, RevBayes, BEAST1 and BEAST2. This will require software that can convert an LPhy specification into an input file that these inference engines understand. The first such software converter is LPhyBEAST described below.

\subsection{LPhyBEAST}

LPhyBEAST is a command-line program that takes an LPhy model specification, and some data and produces a BEAST 2 XML input file.
It is therefore an alternative way to succinctly express and communicate BEAST analyses.


\end{document}
