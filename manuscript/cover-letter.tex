\documentclass{article}

\begin{document}

\noindent Dear PLoS Comp Bio Editor,
\\\\
We are pleased to submit our manuscript titled ``LinguaPhylo: A probabilistic model specification language for reproducible phylogenetic analyses.'' This paper introduces a new domain-specific language, along with its supporting software, which addresses a critical need in the computational biology landscape. Our work focuses on probabilistic modelling of molecular evolution, phylogenetics, and genealogical inference.
\\\\
\noindent Despite significant advancements in the complexity of models applied to comparative molecular datasets, there has been a growing gap in the development of high-level tools that facilitate pedagogy, reproducibility, and reusability of these models. Additionally, there has been limited attention given to high-level formal languages that cater to the simultaneous requirements of precise, compact specifications that are both user-friendly and machine-readable.
\\\\
\noindent Our manuscript presents a solution by introducing a high-level, extensible declarative specification language, featuring implicit typing and array programming capabilities. This language enables concise and accurate representations of state-of-the-art probabilistic models for phylogenetic analyses that are accessible to both humans and computers. We have established a connection between this new language and the widely-used BEAST 2 Bayesian inference tool, as well as developed the LPhy Studio software package for the graphical visualisation and simulation of models specified in the LinguaPhylo language.
\\\\
\noindent We are confident that these developments will be highly relevant and appealing to PLoS Computational Biology's readership.
\\\\
\noindent Sincerely,\\
Alexei Drummond and co-authors

\end{document}
